\documentclass[12pt]{article}
\usepackage{amsmath}
\title{EECS 440: Progamming Assignment 3}
\author{Justin Gray}

\setlength{\parindent}{0pt}
\setlength{\parskip}{1ex plus 0.5ex minus 0.2ex}

\usepackage{graphicx}
\usepackage{float}

\begin{document}

\maketitle

a) Below is the data from the 5 experiments, run with C=1. The data sets were run with 
5-fold cross validation and all results averaged. 
\begin{table}[ht!]
    \begin{tabular}{|c|c|c|c|c|c|} \hline
                             & ab            & cr            & sp              & vo              & ye\\ \hline
    Accuracy ($\mu$,$\sigma$)& 0.785,0.005 & 0.864,0.047 & 0.928,0.004   & 0.947,0.024   &0.696,0.001\\ \hline
    Precision($\mu$,$\sigma$)& 0.795,0.008 & 0.803,0.066 & 0.921,0.007   & 0.930,0.059   &0.00,0.000\\ \hline
    Recall   ($\mu$,$\sigma$)& 0.776,0.004 & 0.939,0.029 & 0.895,0.005   & 0.941,0.053   &0.00,0.000\\ \hline  
    \hline
    \end{tabular}
    \caption{data from the linear SVM Experiments}
\end{table}

\begin{table}[ht!]
    \begin{tabular}{|c|c|c|c|c|c|} \hline
                             & ab            & cr            & sp              & vo              & ye\\ \hline
    Accuracy ($\mu$,$\sigma$)& 0.786,0.006 & 0.865,0.020 & 0.924,0.001   & 0.945,0.029   &0.685,004\\ \hline
    Precision($\mu$,$\sigma$)& 0.797,0.012 & 0.799,0.018 & 0.925,0.006   & 0.930,0.072   &0.455,0.019\\ \hline
    Recall   ($\mu$,$\sigma$)& 0.765,0.005 & 0.939,0.038 & 0.880,0.006   & 0.935,0.008   &0.189,0.020\\ \hline  
    \hline
    \end{tabular}
    \caption{data from the linear SSVM Experiments}
\end{table}


\begin{table}[ht!]
    \begin{tabular}{|c|c|c|c|c|c|} \hline
                             & ab            & cr            & sp              & vo              & ye\\ \hline
    Accuracy ($\mu$,$\sigma$)& 0.586,0.054 & 0.698,0.161 & 0.915,0.008   & 0.820,0.010   &0.689,0.021\\ \hline
    Precision($\mu$,$\sigma$)& 0.660,0.166 & 0.697,0.184 & 0.892,0.011   & 0.689,0.349   &0.489,0.035\\ \hline
    Recall   ($\mu$,$\sigma$)& 0.671,0.256 & 0.803,0.117 & 0.892,0.014   & 0.646,0.346   &0.457,0.065\\ \hline  
    \hline
    \end{tabular}
    \caption{Data from the perceptron experiments}
\end{table}

In every case, both the SVM implementations do as good or better than the perceptron tests. The results between the 
linear svm and the smooth svm are almost identical. Interestingly, the SVM classifiers display a much lower standard 
deviation than the perceptron ones. This indicates that for each fold in the training data the SVM is doing a much better
job of avoiding overfitting. The perceptron training case also included a weight minimization term, but it must 
have been more heavily overshadowed by the loss functin in that case. 

The ssvm and the linear svm show almost identical results, which validates the performance of the smooth svm approximations. 
The only notable exception was on the YE database, where the ssvm performed notably better. It had nearly the same 
accuracy, but much better precision and recall.

b) 

\begin{table}[ht!]
    \begin{tabular}{|c|c|c|c|c|c|} \hline
    C        & vo              & cr  \\ \hline
    0.01     & 0.943,0.025     & 0.864, 0.022 \\ \hline
    2.5075   & 0.947,0.026     & 0.864, 0.022 \\ \hline
    5.005    & 0.940,0.021     & 0.864, 0.022 \\ \hline
    7.5025   & 0.940,0.017     & 0.864, 0.022 \\ \hline
    10       & 0.938,0.020     & 0.864, 0.022 \\ \hline
    
    \hline
    \end{tabular}
    \caption{Accuracy data $(\mu,\ \sigma)$, data from tests on the vo and cr datasets, vs changing values for C}
\end{table}

There was a small, but measurable increase in training times as the value of C increased. The cost of putting more 
weight on mimization of the slack variables must increase the training cost by reducing the magnitude of the 
gradient somehow and causing the optimization to covnerge more slowly. 

The two datasets show very different trends with regard to accuracy as C is increased. Accuracy went up slightly, then 
decreased with increaseing C for the vo data set. For the CR dataset, it did not change at all though. This indicates that
the vo data set is not perfectly linearly seperable, but the cr dataset was. For the linearly seperable case, the 
slack variables will go to zero and the value for C will be meaningless. 


c) 
\begin{table}[ht!]
    \begin{tabular}{|c|c|c|c|c|c|} \hline
                                & ab      & cr    & sp     & vo      & ye\\ \hline
    Decition Tree vs Perceptron & 0.091   & 0.856 & 0.107  & 0.073   &0.204\\ \hline
    Decision Tree vs SVM        & 4.88E-7 & 0.050 & (0.921,0.008)   & 0.481   &0.169\\ \hline
    Perceptron vs SVM           & 7.10e-5 & 0.062 & (0.009,0.010)   & 0.054   &0.511\\ \hline  
    \hline
    \end{tabular}
    \caption{p value from the t-test for the comparison of the different algorithms }
\end{table}

Of all the t-tests run, only a one allows for absolute, difinitive rejection of the null 
hypothosiss at the 95\% confidence level: 

for the ab dataset, the perceptron and the SVM perform better than the Decision Tree. 
It is interesting to note that although the average error rates between the SVM and the Perceptron
differ by over 10\% (favoring the perceptron), you can not distinguish between the accuracy of the two 
from the tests run. 

Of the other tests, the CR dataset comparison between the Decision Tree and the SVM lies exactly 
on the boundary, but the value was roudned down. So you can't actually reject the null hypothesis
in this case. Similarly, the value for the t-test between the Percetron and the SVM for the ye 
dataset is almost low enough (but not quite) to all for the rejection of the null hypothosis. 

Othere than the above cases datasets, the results all very strongly suggest that these tests are not 
complete enough to determine which type of classifier is best on any of these datasets. 
    


\end{document}