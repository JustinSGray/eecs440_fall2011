\documentclass[12pt]{article}
\usepackage{amsmath}
\title{EECS 440: HW 7}
\author{Justin Gray}

\setlength{\parindent}{0pt}
\setlength{\parskip}{1ex plus 0.5ex minus 0.2ex}

\usepackage{graphicx}
\usepackage{float}

\begin{document}

\maketitle
\setcounter{equation}{0}
1) Each network has one trivial case where all variables are conditionally 
independent of eachother. n variables will translate into n nodes in a baysian 
network. 

For any collection of n nodes, there must be $n^2-n-1$ possible different
directed graphs combinations, assuming a fixed order for the nodes. Then, 
reorganizing the nodes will give you n different versions of each combination. 

So the total number of possible networks is $1+ n(n^2-n-1)$. 



\end{document}