\documentclass[12pt]{article}
\usepackage{amsmath}
\title{EECS 440: HW 1}
\author{Justin Gray}

\setlength{\parindent}{0pt}
\setlength{\parskip}{1ex plus 0.5ex minus 0.2ex}

\usepackage{graphicx}
\usepackage{float}

\begin{document}

\maketitle

1)Machine Learning is well suited to the estimation of function values over a 
certain range of parameters, when the cost of obtaining the actual function value
is very high (complex engineering analyses for instance). In this case the performance
measure is usually based on two things; the accuracy of the prediction on the training 
set, and the accuracy of the prediction over an evaluation set. Accuracy is meauserd 
by a percentage difference between the predicted function value and the known value. 
For this use, it is difficult to specify a specifc example representation because it 
is highly dependent on the specific analysis. However, common techniques include 
ploynomial regressions, baysian statitiscal models, and neural networks. 

Looking at a specific example, you could consider the job of estimating the drag on a wing
from a limited number of computational fluid dynamics (CFD) evaluations. The example 
representation is straitforward, you would include the set of parameters which defined
a specific wing design as the attributes. These could be either continous, integer values, 
or possibly discrete (selecting a specific type of NACA airfoil for instance). The 
classifier would be a continous value of drag predicted by the CFD run for each example. 
You would run N such examples throuh the CFD to create a supervised learning set. 
Since the values of drag predicted tend to be highly nonlinear any kind of polynomial 
regression is not a good choice for a hypothosis space. Instead, a neural network or 
kriging model would be more appropriate. The choice between the two would come down to a
few key differences. Firstly, neural networks evaluate much faster than kriging models, 
so for certain analyses, they would be a clear choice (montecarlo simulations or 
pareto optimizations). However, Neural networks also require large training sets. If you
wanted to keep the training cost low, at the expense of evaluation cost of a learned concept, 
you would pick a kriging model. This will have higher accuracy (especially near trained points, 
which it predicts perfectly), and can even return a confidence value for it's prediction. 

In th end, a common solution might be to actually train both types of learning processes
on the same data and come up with some scheme for comparing the two and getting an answer. 

A second interesting use for machine learning would be to predict feeding patterns 
of newborn infants. The training examples would be simple to construct from measurements
of baby phisiology such as temperature, respiration rate, heart rate, weight, and age. 
In addition, you would want to consider time of day, and possibly time since last feeding. 
The class label could be a binary choice of hungry or not-hungry. However, in this case
collecting the data could be somewhat challenging. How do you determine if a baby you're 
measuring is hungry or not hungry? A simple solution would be to simply label them as hungry
if when offered food, they ate. 

In this case, given a binary classification a decision tree might make an appropriate 
choice for a hypothosis space for two reasons. Firstly, there is likely to be a lot of 
noise in the sample set. Obviously, since the "measurements" are imprecise. Also, 
this is an example of where there is likely to be missing data in some of the attributes. 
decision trees handle both those situations well. The training data would be a supervised
learning set of data collected from a large survey of infants. You could not train on a 
single infant, but the result would be rather useless. After a short time, parents can 
become very good at learning this information themselves. The goal would be to come 
up with a perdictor for new parents or for situations (like day-care) where there are a 
lot of babies who could be hungry and it's not a good idea to offer them all food. 

2) For n boolean values, there are $2^n$ possible examples. This is a simple rule for
the number of possible combinations with 2 possible values (binary) for n attributes. 
The number of possible distinct decision trees represents the number of possible 

Each node of a decision tree represents a single binary operation. Either an attribute is
a 1 or a 0. Hence, there are $2^{x}$ possible different combinations of binary operations 
which can be perfomed on $x=2^n$ different possible examples. So the total possible number 
of decision trees is $2^{2^n}$

3) Given that there are some fraction of examples, $p=\frac{m}{k}$, of a set with $f=v$. m is 
the number of cases with $f=v$ and k is the total number of cases in the example set with a value
for f. Now, also suppose that some number, n, of examples have no value for f.

According to the method shown in class the examples with missing data must be 
spread evenly across the bins according the probabilty, p, of an example being in that bin. 
so: 
\begin{equation}
    n_{f=v} = p*n 
\end{equation}

So the number of examples, $m^*$, for which $f=v$ is now: 
\begin{equation}
    m^* = m + n_{f=v} = m+pn = m+\frac{m}{k}n
\end{equation}

The fraction of examples for which $f=v$, $p^*$, is now: 
\begin{equation}
    p^* = \frac{m^*}{n+k} = \frac{m+\frac{m}{k}n}{n+k} = m\frac{1+\frac{n}{k}}{n+k}
    = \frac{m}{k}\frac{k+n}{n+k} = p
\end{equation}

Hence $p^*=p$ and the overall fractions remain unchanged.

\end{document}